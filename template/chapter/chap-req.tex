
\chapter{中国科学院研究生院学位论文撰写要求}
\label{chap:requires}

学位论文是为申请学位而撰写的学术论文,是评判学位申请者学术水平的主要依据,
也是学位申请者获得学位的必要条件之一。为规范和统一我院研究生学位论文的写作,
根据《中华人民共和国学位条例暂行实施办法》的有关规定,提出以下要求:

\section{基本要求}

学位论文必须是一篇(或由一组论文组成的一篇)系统的、完整的学术论文。
学位论文应是学位申请者本人在导师的指导下独立完成的研究成果,
不得抄袭和剽窃他人成果。学位论文的学术观点必须明确,且逻辑严谨,文字通畅。

\subsection{硕士学位论文}

硕士学位论文要注意在基础学科或应用学科中选择有价值的课题,
对所研究的课题有新的见解,并能表明作者在本门学科上掌握了坚实的基础理论和
系统的专门知识,具有从事科学研究工作或独立担负专门技术工作的能力。

硕士学位论文工作一般在硕士生完成培养计划所规定的课程学习后开始,
应包括文献阅读、开题报告、拟定并实施工作计划、科研调查、实验研究、理论分析
和文字总结等工作环节。硕士学位论文必须有一定的工作量。在论文题目确定后,
用于论文工作的时间一般不得少于一年半。

\subsection{博士学位论文}

博士学位论文要选择在国际上属于学科前沿的课题或对国家经济建设和社会发展
有重要意义的课题,要突出论文在科学和专门技术上的创新性和先进性,
并能表明作者在本门学科上掌握了坚实宽广的基础理论和系统深入的专门知识,
具有独立从事科学研究工作的能力。

博士学位论文工作是培养博士学位研究生最重要的环节,其工作时间一般不应少于两年。
博士研究生入学后,要在导师指导下确定科研方向,收集资料,阅读文献,
进行调查研究,选择研究课题。一般在第二学期,最迟在第三学期通过开题报告
并制定论文工作计划,之后根据论文工作计划分阶段报告科研和论文工作进展情况。

\section{学位论文的组成部分和排列顺序}

学位论文一般由以下几个部分组成:封面、论文摘要、论文目录、正文、参考文献、
发表文章目录、致谢等。

\subsection{封面}

根据原国家标准局《科学技术报告、学位论文和学术论文的编写格式》
(国家标准GB7713-87)的封面要求,特规定中国科学院研究生院研究生学位论文的
封面格式(见样张1和样张2),并提出以下具体要求:

\subsubsection{分类号}

必须在封面左上角注明分类号。一般应注明《中国图书资料分类法》的类号,
同时注明《国际十进分类法UDC》的类号。

\subsubsection{编号}

各培养单位自定。

\subsubsection{密级}

论文必须按国家规定的保密条例在右上角注明密级(如系公开型论文则可不注明密级)。

\subsubsection{论文题目}

学位论文题目应当简明扼要地概括和反映出论文的核心内容,一般不宜超过20个字,
必要时可加副标题。

\subsubsection{指导教师}

指导教师必须是被批准上岗的指导教师。

\subsubsection{申请学位级别}

填硕士学位或博士学位。

\subsubsection{学科、专业名称}

按国家颁布的学科、专业目录中的名称填写。

\subsubsection{论文提交日期和论文答辩日期}

按实际提交和答辩日期填写。

\subsubsection{培养单位}

填写培养单位全称。

\subsubsection{学位授予单位}

填写``中国科学院研究生院''。

\subsection{论文摘要}

论文摘要应概括地反映出本论文的主要内容,主要说明本论文的研究目的、内容、
方法、成果和结论。要突出本论文的创造性成果或新见解,不要与引言相混淆。
中文摘要力求语言精炼准确,字数在500字左右。英文摘要内容要与中文摘要内容一致。
并在英文题目下面第一行写研究生姓名。专业名称用括号括起后,置于姓名之后。
研究生姓名下面的一行写导师姓名,格式为:Directed by......。
无论中英文摘要都必须在摘要页的最下方另起一行,注明本文的关键词(3~5个)。

\subsection{论文目录}

论文目录是论文的提纲,也是论文各章节组成部分的小标题。

\subsection{正文}

正文是学位论文的主体和核心部分,不同学科专业和不同的选题可以有不同的写作方式。
正文一般包括以下几个方面:

\subsubsection{引言}

引言是学位论文主体部分的开端,要求言简意赅,不要与摘要雷同或成为摘要的注解。
除了说明研究目的、方法、结果等外,还应评述国内外研究现状和相关领域中已有的研究成果;
介绍本项研究工作前提和任务,理论依据和实验基础,涉及范围和预期结果以及该论文
在已有的基础上所解决的问题。

\subsubsection{各具体章节}

\subsubsection{结论}

结论是学位论文最终和总体的结论,是整篇论文的归宿。应精炼、准确、完整。
着重阐述作者研究的创造性成果及其在本研究领域中的意义,还可进一步提出
需要讨论的问题和建议。

\subsection{参考文献}

学位论文的撰写应本着严谨求实的科学态度,凡有引用他人成果之处,
均应按论文中所引用的顺序列于文末。参考文献的著录均应符合国家有关标准
(按照GB7714-87 《文后参考文献著录格式》执行)。

1.文献是期刊时,书写格式为: 序号\ 作者. 文章题目. 期刊名,
年份(期数):起止页码

2.文献是图书时,书写格式为: 序号\ 作者. 书名. 版次. 出版地:出版单位,年份.
起止页码

\subsection{发表文章目录}

指学位申请者在学期间在各类正式刊物上发表或已被接受的学术论文。

\subsection{致谢}

表达作者对完成论文和学业提供帮助的老师、同学、领导、同事及亲属的感激之情。

\section{学位论文的书写、装订要求}

(一)中国科学院研究生院研究生学位论文必须用中文书写

1. 论文``题目'':黑体小三号

2. 论文``章'':黑体四号

3. 论文``节'':黑体小四号

4. 正文:宋体小四号

5. 为美观方便起见,要有页眉,奇数页上注明每一章名称,偶数页上注明论文题目。

为了便于国际合作与交流,学位论文亦可有英文或其它文字的副本。

(二)文中的图表、附注、参考文献、公式一律采用阿拉伯数字连续(或分章)编号。
如图1,表1,附注:1,文献(1),公式(1)。图序及图名置于图的下方;
表序及表名置于表的上方;论文中的公式编号用括弧括起来写在右边行末,其间不加虚线。

(三)文中所用单位一律采用国务院发布的《中华人民共和国法定计量单位》,
单位名称和符号的书写方式,应采用国际通用符号。

(四)学位论文封面采用全院统一格式,封面用纸为150克花纹纸,博士学位论文
封面颜色为红色,硕士学位论文封面颜色为蓝色(见样张博士学位论文封面、
硕士学位论文封面)。

(五)学位论文一律用A4打印纸装订。
